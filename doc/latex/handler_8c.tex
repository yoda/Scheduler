\hypertarget{handler_8c}{
\section{src/handler.c File Reference}
\label{handler_8c}\index{src/handler.c@{src/handler.c}}
}
{\tt \#include \char`\"{}reader.h\char`\"{}}\par
{\tt \#include \char`\"{}handler.h\char`\"{}}\par
\subsection*{Functions}
\begin{CompactItemize}
\item 
void \hyperlink{handler_8c_65fe8d7216eda7580f414308007e8154}{scheduler} (char $\ast$$\ast$inputfile, char $\ast$$\ast$algorithm, int $\ast$quantum)
\begin{CompactList}\small\item\em Handles the algorithm selection, depending on which scheduling method was specified via the command line. \item\end{CompactList}\end{CompactItemize}


\subsection{Function Documentation}
\hypertarget{handler_8c_65fe8d7216eda7580f414308007e8154}{
\index{handler.c@{handler.c}!scheduler@{scheduler}}
\index{scheduler@{scheduler}!handler.c@{handler.c}}
\subsubsection[{scheduler}]{\setlength{\rightskip}{0pt plus 5cm}void scheduler (char $\ast$$\ast$ {\em inputfile}, \/  char $\ast$$\ast$ {\em algorithm}, \/  int $\ast$ {\em quantum})}}
\label{handler_8c_65fe8d7216eda7580f414308007e8154}


Handles the algorithm selection, depending on which scheduling method was specified via the command line. 

\begin{Desc}
\item[Parameters:]
\begin{description}
\item[{\em inputfile}]The name of the input file containing the processes \item[{\em algorithm}]The desired schedulaing algorithm \item[{\em quantum}]The desired time quantum (for round robin scheduling only) \end{description}
\end{Desc}
